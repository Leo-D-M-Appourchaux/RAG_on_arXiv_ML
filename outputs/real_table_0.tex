\documentclass{article}%
\usepackage[T1]{fontenc}%
\usepackage[utf8]{inputenc}%
\usepackage{lmodern}%
\usepackage{textcomp}%
\usepackage{lastpage}%
\usepackage{float}%
\usepackage{booktabs}%
\usepackage{graphicx}%
\usepackage{multirow}%
\usepackage{makecell}%
\usepackage{cite}%
\usepackage{threeparttable}%
\usepackage{xcolor}%
\usepackage{amssymb}%
\usepackage{hyperref}%
\usepackage{textcomp}%
%
\newcommand{\specialcell}[2][c]{\begin{tabular}[#1]{@{}c@{}}#2\end{tabular}}%
%
\begin{document}%
\normalsize%
\begin{table}[H]
    \centering
    \scriptsize
    \begin{tabular}{|p{3cm}|p{3cm}|p{4cm}|p{3cm}|}
      \hline
      \hline
      \multicolumn{4}{c}{Details of Experiments for the Employed Data Set}\\
      \cline{1-4}
      \emph{Domain} & \emph{Raw Features} & \emph{Response} & \emph{Data Set Cardinality}\\
      \hline
      Australian Credit Scoring & 16 & Desired credit approval of individuals based on characteristics & 690\\\hline
    \end{tabular}
    \caption{\small Data set descriptions for the experiments used to validate the efficacy of the proposed algorithm. We summarize here the domain of the application, the input features to the algorithm, the response variable we wish to predict and the number of examples provided in the data.}
  \end{table}%
\end{document}